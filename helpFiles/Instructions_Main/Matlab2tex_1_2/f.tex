% This file was automatically created from the m-file 
% "m2tex.m" written by USL. 
% The fontencoding in this file is UTF-8. 
%  
% You will need to include the following two packages in 
% your LaTeX-Main-File. 
%  
% \usepackage{color} 
% \usepackage{fancyvrb} 
%  
% It is advised to use the following option for Inputenc 
% \usepackage[utf8]{inputenc} 
%  
  
% definition of matlab colors: 
\definecolor{mblue}{rgb}{0,0,1} 
\definecolor{mgreen}{rgb}{0.13333,0.5451,0.13333} 
\definecolor{mred}{rgb}{0.62745,0.12549,0.94118} 
\definecolor{mgrey}{rgb}{0.5,0.5,0.5} 
\definecolor{mdarkgrey}{rgb}{0.25,0.25,0.25} 
  
\DefineShortVerb[fontfamily=courier,fontseries=m]{\$} 
\DefineShortVerb[fontfamily=courier,fontseries=b]{\#} 
  
\begin{Verbatim}[commandchars=\$\{\},numbers=left,numbersep=2pt] 

    $textcolor{mblue}{function} xdot = f_ex1_2a(x) 
    $textcolor{mgreen}{%--------------------------------------------------------------------------} 
    $textcolor{mgreen}{% Matlab M-file Project: HyEQ Toolbox @  Hybrid Systems Laboratory (HSL), } 
    $textcolor{mgreen}{% https://hybrid.soe.ucsc.edu/software} 
    $textcolor{mgreen}{% http://hybridsimulator.wordpress.com/} 
    $textcolor{mgreen}{% Filename: f_ex1_2a.m} 
    $textcolor{mgreen}{%--------------------------------------------------------------------------} 
    $textcolor{mgreen}{% Project: Simulation of a hybrid system (bouncing ball)} 
    $textcolor{mgreen}{% Description: Flow map} 
    $textcolor{mgreen}{%--------------------------------------------------------------------------} 
    $textcolor{mgreen}{%--------------------------------------------------------------------------} 
    $textcolor{mgreen}{%   See also HYEQSOLVER, PLOTARC, PLOTARC3, PLOTFLOWS, PLOTHARC,} 
    $textcolor{mgreen}{%   PLOTHARCCOLOR, PLOTHARCCOLOR3D, PLOTHYBRIDARC, PLOTJUMPS.} 
    $textcolor{mgreen}{%   Copyright @ Hybrid Systems Laboratory (HSL),} 
    $textcolor{mgreen}{%   Revision: 0.0.0.3 Date: 05/20/2015 3:42:00} 
     
    $textcolor{mgreen}{% state} 
    x1 = x(1); 
    x2 = x(2); 
     
    $textcolor{mblue}{global} gamma 
     
    $textcolor{mgreen}{% differential equations} 
    xdot = [x2 ; gamma]; 
    $textcolor{mblue}{end}  
\end{Verbatim}  
  
\UndefineShortVerb{\$} 
\UndefineShortVerb{\#} 
 