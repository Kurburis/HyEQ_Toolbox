% This file was automatically created from the m-file 
% "m2tex.m" written by USL. 
% The fontencoding in this file is UTF-8. 
%  
% You will need to include the following two packages in 
% your LaTeX-Main-File. 
%  
% \usepackage{color} 
% \usepackage{fancyvrb} 
%  
% It is advised to use the following option for Inputenc 
% \usepackage[utf8]{inputenc} 
%  
  
% definition of matlab colors: 
\definecolor{mblue}{rgb}{0,0,1} 
\definecolor{mgreen}{rgb}{0.13333,0.5451,0.13333} 
\definecolor{mred}{rgb}{0.62745,0.12549,0.94118} 
\definecolor{mgrey}{rgb}{0.5,0.5,0.5} 
\definecolor{mdarkgrey}{rgb}{0.25,0.25,0.25} 
  
\DefineShortVerb[fontfamily=courier,fontseries=m]{\$} 
\DefineShortVerb[fontfamily=courier,fontseries=b]{\#} 
  
\begin{Verbatim}[commandchars=\$\{\},numbers=left,numbersep=2pt] 

    $textcolor{mblue}{function} xplus = g(x, vs, Tnmax, Tnmin, tk) 
    $textcolor{mgreen}{%--------------------------------------------------------------------------} 
    $textcolor{mgreen}{% Matlab M-file Project: HyEQ Toolbox @  Hybrid Systems Laboratory (HSL), } 
    $textcolor{mgreen}{% https://hybrid.soe.ucsc.edu/software} 
    $textcolor{mgreen}{% http://hybridsimulator.wordpress.com/} 
    $textcolor{mgreen}{%--------------------------------------------------------------------------} 
    $textcolor{mgreen}{% Project: Simulation of a hybrid system Analog-to-Digital converter (ADC) } 
    $textcolor{mgreen}{% Description: Jump map} 
    $textcolor{mgreen}{%--------------------------------------------------------------------------} 
    $textcolor{mgreen}{%--------------------------------------------------------------------------} 
    $textcolor{mgreen}{%   See also HYEQSOLVER, PLOTARC, PLOTARC3, PLOTFLOWS, PLOTHARC,} 
    $textcolor{mgreen}{%   PLOTHARCCOLOR, PLOTHARCCOLOR3D, PLOTHYBRIDARC, PLOTJUMPS.} 
    $textcolor{mgreen}{%   Copyright @ Hybrid Systems Laboratory (HSL),} 
    $textcolor{mgreen}{%   Revision: 0.0.0.3 Date: 05/20/2015 3:42:00} 
     
    n = length(vs); $textcolor{mgreen}{% measured input size} 
    xtemp = zeros(n+2,1); 
    xtemp = x; 
    x = xtemp; 
     
     
    j = x(n+1); 
    msplus = vs; $textcolor{mgreen}{% output = measured input} 
    $textcolor{mgreen}{% The value tau_s is updated as a function of vs, e.g., } 
    tau_splus = tk(j+1); $textcolor{mgreen}{% Timer tau_s} 
    j_plus = j+1; 
    xplus = [msplus;j_plus;tau_splus]; 
     
     
      
\end{Verbatim}  
  
\UndefineShortVerb{\$} 
\UndefineShortVerb{\#} 
 